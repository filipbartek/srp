\documentclass{article}
\usepackage[utf8]{inputenc}
\usepackage[T1]{fontenc}
\usepackage[english]{babel}

\usepackage{amsfonts} % mathbb
\newcommand{\nn}{\mathbb{N}_n}

\newcommand{\file}[1]{\texttt{#1}}
\newcommand{\code}[1]{\texttt{#1}}
\newcommand{\clpfd}{\texttt{clpfd}}
\newcommand{\prolog}{Prolog}
\newcommand{\sicstusprolog}{SICStus \prolog{}}

% Required by glossaries
% http://tex.stackexchange.com/a/93770
\usepackage{datatool}

\usepackage{hyperref}

\usepackage{glossaries}
\makeglossaries
\begin{document}
\title{Stable roommates problem solver}
\author{Filip Bártek}
\maketitle

\section{Problem definition}

Let's have $n$ participants.
Each participant knows some of the participants,
let's call these her potential partners.
Each participant has a linear ordering of her potential partners according
to preference.

Note that a participant may or may not consider herself a potential partner,
i.e. the relation of potential partnership needn't be irreflexive.

A matching is an equivalence relation on participants that has classes of size
at most 2,
i.e. assigns each participant one or none partner.
Matching must assign a potential partner to each of the participants.

An instability in a matching is a pair of participants each of whom prefers (according to their personal preference relations) the other to their current partner.

A stable matching is a matching that doesn't admit an instability.

In stable roommates problem, given preferences of each participant,
the task is to find a stable matching.

\subsection{Perfect matching}
A perfect matching is a matching in which every participant is assigned somebody else.

Once we can solve general SRP,
we can force a perfect matching by making sure that no participant considers herself
a potential partner.

\section{Constraint model}
In this section we'll describe a model for SRP instance on $n$ participants.

We'll use the symbol $\nn$ to denote the set $\{1, \ldots, n\}$.
Each participant is uniquely identified by a number from $\nn$.

Note that the choice of variables and constraints corresponds to the capabilities
of \clpfd{}, a library that is used prominently in the implementation
of the solver.

\subsection{Variables}
\paragraph{Problem instance}
We'll represent an instance of SRP with $n$ participants as
a collection of preference lists $P = (P_1, \ldots, P_n)$.
$P_i$ is a preference list expressing preferences of participant $i$.
It's a $k_i$-tuple of participant identifiers (i.e. numbers from $\nn$).
All potential partners of participant $i$ are listed in $P_i$ in order of decreasing
desirability without duplicities.

\paragraph{Problem solution}
$m: \nn \rightarrow \nn$ assigns each participant her partner.

\paragraph{Auxiliary variables}
Let $s: \nn \times \nn \rightarrow \nn$ be a score function.
$s(i,j)$ represents how desirable participant $j$ is according to participant $i$.
The lower the score, the more desirable participant $j$ is.

$s$ is defined uniquely for a given instance $P$.
For every participant $i$:

\begin{itemize}
\item $j$-th potential partner is assigned score $j$
\item every participant which is not a potential partner is assigned a score $n$
\end{itemize}

As a convenience, let $\bar{s}(i)$ denote the score that participant $i$ assigns to her partner.

\subsection{Constraints}
Since the score function $s$ depends uniquely and trivially
to the problem instance $P$,
construction of $s$ from $P$ is realized
using standard \prolog{} code (without the use of \clpfd{}) and as such,
I'll leave out formal definition of the corresponding constraint.

Similarly, the properties of $P$ are not enforced formally so I'll leave out their
formal definitions.
Informal description of these properties are available in the previous section.

The following constraints constrain the solution $m$ based on $P$ and $s$:

\begin{tabular}{l | l}
Constraint & Explanation \\
\hline
$\forall i. m(i) \in P_i$ & Matching satisfies potential partners \\
$\forall i,j. m(i) = j \Leftrightarrow m(j) = i$ & Matching is symmetric \\
$\forall i. \bar{s}(i) = s(i, m(i))$ & $\bar{s}$ corresponds to $s$ and $m$ \\
$\forall i,j. \bar{s}(i) \leq s(i,j) \vee \bar{s}(j) \leq s(j,i)$ & All pairs are stable
\end{tabular}

Note especially the last of these constraints
as it captures the essence of the problem, that is the requirement of stability.
An instability occurs when a pair of participants prefer each other to their partners.
The constraint ensures that in each pair of participants,
at least one of them prefers her partner (i.e.~assigns her a lower score)
to the other participant.

\section{Implementation}
I've implemented the constraint model introduced in the previous section
using \sicstusprolog{} and its \clpfd{} library.
The implementation is available in the attached file \file{srp.pl}.

\subsection{Usage}
The main entry point is the predicate \code{srp/2}.
It's typical usage is \code{srp(Preferences, Matching)},
where \code{Preferences} is a list of ordered lists of potential partners
and \code{Matching} is unbound.
If a solution of SRP instance described by \code{Preferences} exists,
it is unified with \code{Matching}.
Otherwise \code{srp} fails.

\subsection{Performance}
I've examined the performance of the solver using various problem instances
(esp.~of various sizes) and search strategies.

\subsubsection{System configuration}
The measurements were performed on a computer with the following configuration:

\begin{tabular}{| l | l |}
\hline
Model & HP Compaq 6510b \\
\hline
CPU & Intel Core2 Duo T8100 2.10 GHz \\
\hline
RAM & 2.50 GB \\
\hline
Operating system & Windows Vista Business 32-bit \\
\hline
\sicstusprolog{} & 4.2.3 \\
\hline
\end{tabular}

\subsubsection{Problem instance generation}
I ran the solver on randomly generated problem instances.
For a given $n$, each of the participants' preference lists
is a random permutation of $\nn$.
Every possible permutation occurs with equal probability.

\subsubsection{Measurement procedure}
For every examined value of $n$, 10 random problem instances were generated.
Each of these instances was solved using each of the three search strategies:

\begin{itemize}
\item \code{leftmost},
\item \code{ff} (first-fail) and
\item \code{ffc} (most constrained).
\end{itemize}

Two measurements were performed: assumptions and time.
%TODO: Clarify.

In all cases, the solver was only run until it found the first solution.
In cases the solver couldn't find any solution,
no measurements were performed for that instance.

For time measurement, the solver was run 100 times on each instance;
the reported times are averages from these 100 executions.

\subsubsection{Results}

%TODO: Finish.

\printglossaries{}

\end{document}
